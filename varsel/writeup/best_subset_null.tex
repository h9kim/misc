\title{Computing the null distribution for best-subset}
\author{Charles Zheng and Trevor Hastie}
\date{\today}

\documentclass[12pt]{article} 

% packages with special commands
\usepackage{amssymb, amsmath}
\usepackage{epsfig}
\usepackage{array}
\usepackage{ifthen}
\usepackage{color}
\usepackage{fancyhdr}
\usepackage{graphicx}
\usepackage{mathtools}
\usepackage{csquotes}
\definecolor{grey}{rgb}{0.5,0.5,0.5}

\begin{document}
\maketitle

\newcommand{\tr}{\text{tr}}
\newcommand{\E}{\textbf{E}}
\newcommand{\diag}{\text{diag}}
\newcommand{\argmax}{\text{argmax}}
\newcommand{\Cov}{\text{Cov}}
\newcommand{\Var}{\text{Var}}
\newcommand{\argmin}{\text{argmin}}
\newcommand{\Vol}{\text{Vol}}
\newcommand{\comm}[1]{}

\newcommand{\bx}{\boldsymbol{x}}
\newcommand{\by}{\boldsymbol{y}}
\newcommand{\bX}{\boldsymbol{X}}
\newcommand{\bY}{\boldsymbol{Y}}


\section{Introduction}

Given $n \times p$ design matrix $X$ and data $y$, the \emph{best
k-subset} procedure finds a subset $S \subset \{1,..,p\}$ of size $k$
which maximizes the coefficient of determination, $R^2$:
\[
R^2(S) := \frac{||P_S y||^2}{||y||^2}
\]
where
\[
P_S = X_S (X_S^T X_S)^{-1} X_S^T
\]
and $X_S$ is the submatrix $X$ with columns indexed by $S$.

The best k-subset coefficient of determination is defined
\[
R^2_k = \sup_{|S| = k} R^2(S).
\]

In this work we consider computing the \emph{null distribution} of
$R^2_k$ under the null hypothesis that the data is pure Gaussian
noise: $y \sim N(0, \sigma^2 I)$.

A potential application of this work is testing the null hypothesis
versus the alternative hypothesis that the data was generated from a
sparse linear model.

\section{Intersection-Union method}

Define $Q_S$ as the $Q$ matrix obtained from the QR-decomposition of
$X_S$.  We have
\[
||P_S y||^2 = ||Q_S y||^2
\]
so we can also write
\[
R^2(S) = \frac{||Q_S y||^2}{||y||^2}.
\]

Let $\mathcal{S}$ denote a set of subsets of $\{1,...,p\}$.  For instance, for best-$k$ subset, we would take
\[
\mathcal{S} = \{S \subset \{1,...,p\}: |S| = k\}
\]
but the theory may also be applied to more general families of
subsets.

Define
\[
R^2(\mathcal{S}) = \max_{S \in \mathcal{S}} R^2(S) = \max_{S \in \mathcal{S}} R^2(S) \frac{||Q_S y||^2}{||y||^2}.
\]

We would like to compute the exceedence probability
$\Pr[R^2 \geq \tau]$ when $y \sim N(0, I)$, for arbitrary $\tau \in
[0, 1]$.

The intersection-union formula gives
\begin{align*}
\Pr[R^(\mathcal{S}) \geq \tau] &= \Pr[\cup_{S \in \mathcal{S}} R^2(S) \geq \tau] 
\\&= \sum_{j=1}^{|\mathcal{S}|} (-1)^{j+1} \sum_{S_1 \neq ... \neq S_j \in \mathcal{S}} \Pr[\min_i R^2(S_i) \geq \tau] 
\\&= \sum_{S \in |S|} \Pr[R^2(S) \geq \tau] - \sum_{S_1 \neq S_2} \Pr[R^2(S_1)\vee R^2(S_2) \geq \tau] + ...
\end{align*}

If we could approximate the most important terms in the expansion,
then we could approximate the exceedence probability
$\Pr[R^2(\mathcal{S}) \geq \tau]$.  We present empirical evidence that
the first two terms suffice in the subsection, ``How many terms are
needed?''

\subsection{How many terms are needed?}

\end{document}



