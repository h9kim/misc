

\documentclass[12pt]{article} 

% packages with special commands
\usepackage{amssymb, amsmath}
\usepackage{epsfig}
\usepackage{array}
\usepackage{ifthen}
\usepackage{color}
\usepackage{fancyhdr}
\usepackage{graphicx}
\usepackage{mathtools}
\usepackage{csquotes}
\usepackage{chngcntr}
\usepackage{apptools}
\AtAppendix{\counterwithin{lemma}{section}}

\definecolor{grey}{rgb}{0.5,0.5,0.5}

\begin{document}

7. The formula for the misclassification risk is
\[
\text{Risk} = \sum_{k = 1}^K \sum_{\ell = 1}^K \Pr[y = c_\ell, \hat{y} = c_k] L_{kl}.
\]
According to the problem, we can assume that $L_{kl} = r$ for some constant $r > 0$ for all $k \neq \ell$. Furthermore, we know that $L_{kk} = 0$ because the loss is zero for incorrect classification.  Using these two facts, we get
\[
\text{Risk} = \sum_{k = 1}^K \sum_{\ell = 1}^K \Pr[y = c_\ell, \hat{y} = c_k] r 1(k \neq l)
\]
\[
 = r \sum_{k = 1}^K \sum_{\ell = 1}^K \Pr[y = c_\ell, \hat{y} = c_k] 1(k \neq l)
\]
\[
 = r \Pr[y \neq \hat{y}].
\]
which is a constant times the error rate.

In the general case, the Bayes prediction rule for a new observation $x$ is to predict $\hat{y} = c_k$, where
\[
k = \text{argmin}_k \sum_{\ell=1}^{K} \Pr(y=c_\ell|x) L_{kl}.
\]

However, if $L_{kl} = 1(k \neq \ell)$, then the above simplifies to
\[
k = \text{argmax}_k \Pr(y = c_k|x),
\]
or equivalently,
\[
k = \text{argmax}_k p(x|y=c_k) \pi(y).
\]
In other words, the Bayes prediction rule chooses the class with the highest posterior probability in the special case of zero-one loss.

16. The advantages of such a model include:
\begin{itemize}
\item Invariance to linear transformations of the input
\item May in some situations be a better fit to the true function
\end{itemize}
However, some disadvantages are:
\begin{itemize}
\item Increases the order of the computation in finding the best split from order $O(p)$ to order exponential in $p$.
\item Less fine-grained control over bias-variance tradeoff, because the complexity increases greatly at each split.
\item More difficult to come up with a strategy for dealing with missing variables, since surrogate splitting strategy cannot be applied.
\item More difficult to interpret the resulting trees.
\end{itemize}

Fun fact: (not expected as part of the answer.)  Such trees are known as ``oblique decision trees'' or decision trees with linear decision rules.  They were included in the book that introduced CART ( Breiman, L., Friedman, J., Stone, C. J., Olshen, R. A., 1984. Classification and regression trees. CRC press.), and have been revisited periodically in the machine learning literature since then.  But they do not appear to be widely used in applications.



\end{document}





