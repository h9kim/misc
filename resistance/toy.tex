\documentclass[11pt]{article} 

% packages with special commands
\usepackage{amssymb, amsmath}
\usepackage{epsfig}
\usepackage{array}
\usepackage{ifthen}
\usepackage{color}
%\usepackage{fancyhdr}
\usepackage{graphicx}
\usepackage{indentfirst}
\usepackage{caption}
%\usepackage{mathtools}
\definecolor{grey}{rgb}{0.5,0.5,0.5}

\begin{document}
\newcommand{\tr}{\text{tr}}
\newcommand{\E}{\textbf{E}}
\newcommand{\diag}{\text{diag}}
\newcommand{\argmax}{\text{argmax}}
\newcommand{\argmin}{\text{argmin}}
\newcommand{\Cov}{\text{Cov}}
\newcommand{\Vol}{\text{Vol}}
%\pagestyle{fancy}

\title{A toy model for voting in the Resistance}

\author{Charles Zheng}


\maketitle

\section{Introduction}

Resistance is a bluffing game where players have hidden roles.  There
are two teams: resistance, and spies.  Players are secretly assigned
to one of the two teams, with the majority of the players being
assigned to the resistance, and are privately revealed their
assignment.  It is a zero-sum game between the two teams.  The
mechanics of the game are complicated and somewhat arbitrary, so we
give an abstracted and oversimplified description.  The object of the
game for the resistance is for the every member of the resistance to
correctly guess the identities of the spies at the end of the game.
Since various events in the game leak information about the identities
of the spies, the goal of the spies is to end the game before the
resistance players can discover all of their identities.  Each player
takes actions at various points in the game which either advance the
game towards ending (and therefore spy victory), or which reveal
information about the players' identities, or both.  Some actions are
publically observed, while others are concealed to some degree.  The
spies have private information which allows them to know how much any
particular action advances the game towards ending, but a resistance
player only has private and partial information about how much any
particular action made by themselves or another player advances the
game towards ending.

If I am a resistance player, and I observe you to take many actions
which appear to be advancing the game towards ending, then I will
start to suspect you to be on the spy team.  If the spies are too
aggressive in taking such actions, then the information held by the
resistance will accumulate, and the probability increases that the
resistance will be able to correctly identify the spies.  Furthermore,
if the game progresses too long, spies are eventually forced to take
increasingly suspicious actions, so there is no way a spy can avoid
drawing suspicion to themselves.

The full dynamics of the game involve an interplay of strategy,
teamwork, interrogation and complex conceits for players of both
teams, and one where the reputation of the individual players cannot
be disentangled from the individual games.  It is an extremely human
game which cannot be easily studied using simplified models.  On the
other hand, even without the human elements, the strategic aspects of
the game remain nontrivial.  The question we intend to investigate
here is: if we exclude the human factor, how much information can be
gained from the actions of the players under optimal play?  To answer
this question we introduce a toy model of the game and study its
properties.  The toy model has the minimal level of complexity to
capture the fact that players take actions which either benefit or
hurt the resistance team, and which also reveal information about
their hidden roles.

\section{Noisy votes model}

The game includes one special player, a judge $J$, and $N$ other
players, called \emph{voters}.  $N_G$ of the voters are on the Good
team and $N_B$ of the voters are on the Bad team.  The voters are
randomly assigned, so that each individual voter has an $N_G/N$
probability of being on the Good team.  Let $G_1,\hdots, G_{N_G}$
denote the Good voters and $B_1,\hdots,B_{N_B}$ denote the Bad voters.

The game consists of a voting round and a judging round.  In the
voting round, the voters simultaneously and private choose a vote $V
\in \{-1, 1\}$. Each Good voter $G_i$ independently chooses action
$V_{G_i}=1$, or `upvote', with probability $1-\epsilon$ and action
$V_{G_i}=0$, or `downvote' with probability $\epsilon$.  Each Bad
voter $B_j$ independently chooses to upvote, $V_{B_j}=1$ with
probability $1-p$, and downvote, $V_{B_j}=0$, with probability $p$.
The ``noisy vote'' probability $\epsilon$ is fixed; however, the Bad
voters can collectively choose $p \in [0,1]$ before each voting round.
Let $S$ be the sum of the upvotes,
\[
S =V_{G_1} + \cdots + V_{G_{N_G}} + V_{B_1} + \cdots + V_{B_{N_G}}
\]
and $D = N-S$ be the number of downvotes.  Also define $D_B$ as the
total number of downvotes from the Bad voters, and $D_G$ the total
number of downvoters from the Good voters.

In the judging round, the judge $J$ guesses the identities of the Bad
voters.  The judge is required to pick $L$ players to identify as
Bad players; let $C$ be the number of those players who are correctly
identified.

The judge and the Good voters are on the same team, and their payoff
$P(S, C)$ is a function of the number of upvotes $S$ and the number of
Bad voters correctly identified by the judge, $C$.  The game is
zero-sum, so the payoff for the Bad votes is $-P(S, C)$.

To simplify the analysis, we let Nature control the actions of the Judge, so $J$ no longer counts as an actual ``player''.  Now the action of the judge is prescribed to be as follows:
\begin{itemize}
\item If there are more than $L$ downvoters, choose $L$ of the players amongst the downvoters uniformly at random
\item Otherwise, choose all $D$ of the downvoters, and choose $L - D$ of the players amongst the upvoters uniformly at random
\end{itemize}

The form of the payoff function $P(S,C)$ and the parameters $N_B, N_G$
and $\epsilon$ determine the set of optimal choices of downvoting
probability $p$ for the Bad players.  A particular optimal choice of
$p$ determines an equilibrium strategy for the game.  We categorize
the equilibria as follows:
\begin{enumerate}
\item A unique equilibrium with $p \in [0, \epsilon]$
\item A unique equilibrium with $p = 1$
\item A unique equilibrium with $p \in (\epsilon, 1)$
\item Every $p \in [0,1]$ results in the same expected payoff, hence
  there is no unique eqilibrium.
\end{enumerate}

In equilibrium 1, Bad voters are statistically indistinguishable from
Good voters, and hence we conclude that there is no information to be
gained from the votes.

In equilibrium 2, $p=1$, votes contain the maximal amount possible
under the model.  However, since the best strategy for spies is
deterministic, this is an uninteresting case.

In equilibrium 3, votes contain information and the strategy is
non-degenerate.  Hence we might be most interested in model parameters
which produce this class of equilibria.

Equilibrium 4 can occur if the payoff function is chosen poorly.  For
all practical purposes, spies could just choose $p=0$, and we conclude
that no information can be obtained.

\subsection{Simple case}

A particularly simple case of the model is if $L=1$ and
\[
P(S,C) = \alpha I(S = N) + \beta I(C = 1)
\]
The Good team gets a reward if there are no downvotes, and another
reward if they guess a Bad voter correctly.

The expected reward (or value of the game) $V$ can be decomposed as
\[
V = \E[P(S,C)] = \alpha \Pr[S = N] + \beta \Pr[C=1]
\]

Then,
\[
\Pr[S = N] = (1-\epsilon)^{N_G}(1-p)^{N_B}
\]
while
\[
\Pr[C = 1] = \E\left[\frac{I(D \geq 1)D_B}{D}\right] + \frac{N_B}{N_G}\Pr[S = N]
\]
Let us define $f(p,\epsilon) = \Pr[C=1]$ and define $\tilde{\alpha} = (\alpha + \beta\frac{N_B}{N_G})(1-\epsilon)^{N_G}$.
Then we have
\[
V = \E[P(S,C)] = \tilde{\alpha} (1-p)^{N_B} + \beta f(p, \epsilon)
\]

It is not possible to write a closed-form expression for
$f(p,\epsilon)$, but we can try to derive some useful approximations.
We know that $S_B \sim \text{Binomial}(N_B, p)$ and $S_G \sim
\text{Binomial}(N_G, \epsilon)$, with $S_B$ and $S_G$ independent.








\end{document}
