\documentclass[11pt]{article} 

% packages with special commands
\usepackage{amssymb, amsmath}
\usepackage{epsfig}
\usepackage{array}
\usepackage{ifthen}
\usepackage{color}
%\usepackage{fancyhdr}
\usepackage{graphicx}
\usepackage{indentfirst}
\usepackage{caption}
%\usepackage{mathtools}
\definecolor{grey}{rgb}{0.5,0.5,0.5}

\begin{document}
\newcommand{\tr}{\text{tr}}
\newcommand{\E}{\textbf{E}}
\newcommand{\diag}{\text{diag}}
\newcommand{\argmax}{\text{argmax}}
\newcommand{\argmin}{\text{argmin}}
\newcommand{\Cov}{\text{Cov}}
\newcommand{\Var}{\text{Var}}
\newcommand{\Vol}{\text{Vol}}
\newcommand{\HH}{\boldsymbol{H}}
%\pagestyle{fancy}

\title{Minimum coefficient of variation for log-concave densities}

\author{Charles Zheng}


\maketitle

\section{Problem}

Let $f(x)$ be a twice-differentiable convex function on $\mathbb{R}^+$ such that $\int_0^\infty e^{-f(x)} < \infty$.
Then, if $X$ is distributed according to
\[
\Pr[X < t] = \frac{\int_0^t e^{-f(x)} dx}{\int_0^\infty e^{-f(x)} dx}
\]
we say that $X$ has a log-concave distribution on the positive real line.

Define the coefficient of variation of $X$ by
\[
\text{CV}[X] = \frac{\E[X]}{\sqrt{\Var[X]}}.
\]

Accordingly, define functionals
\[
E(f) = \frac{\int_0^\infty x e^{-f(x)} dx}{\int_0^\infty e^{-f(x)} dx}
\]
\[
E_2(f) = \frac{\int_0^\infty x^2 e^{-f(x)} dx}{\int_0^\infty e^{-f(x)} dx}
\]
\[
V(f) = E(f)^2 - E_2(f)
\]
\[
\text{CV}(f) = \frac{E(f)}{\sqrt{V(f)}}
\]
so that $E(f) = \E[X]$, $E_2(f) = \E[X^2]$, $V(f) = \Var(X)$ and $\text{CV}(f) = \text{CV}[X]$ for $X$ defined as above.

Now consider the problem of finding the log-concave distribution with the smallest coefficient of variation, i.e.
\[
\text{minimize}_f \ \text{CV}(f) \text{ subject to }f'' \geq 0.
\]

Intuitively, if the above optimization has a unique solution, then it
should lie on the boundary of the constraint, hence $f''(x) = 0$.
This suggests the exponential distribution, corresponding to $f(x) =
x/\lambda$, which has a coefficient of variation equal to 1.  In the
following, we will use the variational calculus to show that $f(x) =
x$ is a local minimum of the optimization problem, but we do not have
a proof that $f(x) = x$ is a global minimum.

\section{Calculus of Variations}

The calculus of variations allows one to define gradients of functionals.
Given a functional $\Lambda : \mathbb{F} \to \mathbb{R}$ with function space $\mathcal{F}$ as domain , 
any function $h$ in the dual space $\bar{\mathcal{F}}$
 such that
\[
\lim_{\epsilon \to 0} \frac{\Lambda(f + \epsilon g) - \Lambda(f)}{\epsilon} = \int h(x) g(x) dx
\]
for all $g \in \mathcal{F}$
is called a gradient of $\Lambda$ at $f$:
\[h = \nabla \Lambda(f) .\]

In our problem, the gradients of the functionals $E$, $V$, and $\text{CV}$ are given as follows:
\[
\nabla E(f) = p_f(x) (E(f) - x)
\]
\[
\nabla V(f) = p_f(x) (E_2(f) - 2E(f)^2 + 2 E(f) x - x^2)
\]
\[
\nabla \text{CV}(f) = p_f(x) \left[
\frac{E(f) - x}{\sqrt{V(f)}} - \frac{E(f)}{2V(f)^{3/2}}\left(
E_2(f) - 2E(f)^2 + 2 E(f) x - x^2
\right)
\right].
\]
where
\[
p_f(x) = \frac{e^{-f(x)}}{\int_0^\infty e^{-f(z)} dz}.
\]

Letting $h_f = \nabla \text{CV}(f)$, in order to show that $f$ is a local minimum, it suffices to show that
\[
\lim_{\epsilon \to 0} \frac{1}{\epsilon}(\text{CV}(f + \epsilon g) - \text{CV}(f)) > 0,
\]
i.e. that
\[
\int_0^\infty h_f(x) g(x) \geq 0
\]
for any $g$ such that $g''(x) \geq 0$ for all $x$ such that $f''(x) = 0$.

Define $H_f(x) = \int_0^x h_f(z) dz$ and $\HH (x) = \int_0^x H_f(z) dz$.
Then we for any $t > 0$, applying integration by parts we have
\[
\int_0^t h_f(x) g(x) dx = H_f(x) g(x) |_0^t - \HH_f (x) g'(x)|_0^t + \int_0^t \HH_f(x) g''(x) dx.
\]
Hence, supposing that $H_f(0) = 0$, $\HH_f(0) = 0$, and
\[
0 = \lim_{x \to \infty} H_f(x) g(x) = \lim_{x \to \infty} \HH_f (x) g'(x)
\]
we have
\[
\int_0^\infty h_f(x) g(x) = \int_0^\infty \HH_f(x) g''(x).
\]

This motivates the following lemma:

\noindent\textbf{Lemma}. \emph{
Suppose that 
\[0 = \lim_{x \to \infty} x H_f(x) = \lim_{x \to \infty} \HH_f(x),\]
and also that there exists $x^* > 0$ such that $\inf_{x \geq x^*} h_f(x) \geq 0$.
Then
\[
\inf_{g: g'' \geq 0}\ \int_0^\infty g(x) h_f(x) dx < 0.
\]
implies
\[
\inf_{g: g'' \geq 0}\ \int_0^\infty g''(x) \HH_f(x) dx < 0
\]
}

\noindent \textbf{Proof.}  
Suppose that
\[
\inf_{g: g'' \geq 0}\ \int_0^\infty g(x) h_f(x) dx < 0.
\]
Then there exists $g$ with $g'' \geq 0$ such that
\[
\int_0^\infty g(x) h_f(x) dx  = \delta < 0.
\]
Now consider piecewise functions $\tilde{g}$ of the form
\[
\tilde{g}(x) = \begin{cases}
g(x) &\text{ for } x < t^*\\
j(x) &\text{ for } x \in [t^*, t^* + \epsilon]\\
j(t^* + \epsilon) + (x-t^*)g'(t^* + \epsilon) &\text{ for }x \geq t^*
\end{cases}
\]
where $j$ is chosen so that $j(x) \leq g(x)$ and so that $\tilde{g}$ is twice-differentiable and convex,
Since $\tilde{g}(x) \leq g(x)$ for $x \geq t^*$,
and also since $h_f(x) \geq 0$ for $x \geq t^*$,
we conclude that $\tilde{g}(x) h_f(x) \leq g(x) h_f(x)$ pointwise,
and hence
\[
\int_0^\infty \tilde{g}(x) h_f(x) dx \leq \int_0^\infty g(x) h_f(x) < 0.
\]
Furthermore, we can modify $\tilde{g}$ to be twice-differentiable while preserving the above property.
Hence we conclude that there exists
$\tilde{g}$ with
\[
\int_0^\infty \tilde{g}(x) h_f(x) dx  < 0.
\]
and also
\[
0 = \lim_{x \to \infty} H_f(x) \tilde{g}(x) = \lim_{x \to \infty} \HH_f(x) \tilde{g}'(x).
\]
This allows us to conclude that
\[
\int_0^\infty \tilde{g}(x) h_f(x) dx = \int_0^\infty \tilde{g}''(x) \HH_f(x) dx < 0,
\]
completing the proof. $\Box$

\subsection{Exponential distribution}

For $f(x) = x$, the exponential distribution, we have
\[
h_f(x) = \frac{1}{2}e^{-x}(x^2 - 4x + 2)
\]
\[
H_f(x) = \frac{1}{2}e^{-x}(2x-x^2)
\]
\[
\HH_f(x) = \frac{1}{2}e^{-x}(x^2 + 4x).
\]

Since $\HH_f(x) \geq 0$, it is clear that
\[
\inf_{g: g'' \geq 0}\ \int_0^\infty \HH_f(x) g''(x) dx \geq 0.
\]
Now since $H_f(x)$ and $\HH_f(x)$ satisfy the conditions of the lemma, we can conclude from the contrapositive of the lemma that
\[
\inf_{g: g'' \geq 0}\ \int_0^\infty h_f(x) g(x) dx \geq 0,
\]
which thus implies that $f(x) = x$ is a local minimum.


\end{document}
