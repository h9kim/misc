\documentclass[11pt]{article} 

% packages with special commands
\usepackage{amssymb, amsmath}
\usepackage{epsfig}
\usepackage{array}
\usepackage{ifthen}
\usepackage{color}
\usepackage{fancyhdr}
\usepackage{graphicx}
%\usepackage{mathtools}
\definecolor{grey}{rgb}{0.5,0.5,0.5}

\begin{document}
\newcommand{\tr}{\text{tr}}
\newcommand{\E}{\textbf{E}}
\newcommand{\diag}{\text{diag}}
\newcommand{\argmax}{\text{argmax}}
\newcommand{\argmin}{\text{argmin}}
\newcommand{\Cov}{\text{Cov}}
\newcommand{\Vol}{\text{Vol}}
\pagestyle{fancy}

\title{Semi-supervised Principal Components Regression}

\author{Charles Zheng}

\maketitle

\begin{abstract}
\emph{Semi-supervised learning} refers to the problem of learning a
rule for predicting labels $y$ from features $x$, given training data
which includes both labeled examples $(x_1, y_1), \hdots, (x_\ell,
y_\ell)$ and unlabeled examples $x_{\ell + 1}, \hdots, x_n$.  A basic
question of this field is to characterize the conditions under which
the unlabeled examples can be used to improve generalization error.
We introduce a latent variable model for which semi-supervised
principal components regression is shown to outperform supervised
ridge regression.\\

Keywords: Semi-supervised, ridge regression,
principal components analysis, latent variables
\end{abstract}

\section{Introduction}

\subsection{Ridge Regression}

Ridge regression is a linear method for predicting a real-valued label
$y \in \mathbb{R}$ from a vector of real-valued features $x \in
\mathbb{R}^p$.  It is produces a regularized least squares estimate
for a coefficient vector $\hat{\beta}_\lambda$ and intercept term
$\hat{c}_\lambda$, where $\lambda$ is a regularization parameter.
This coefficient vector $\hat{\beta}_\lambda$ is used to predict the
label $y^*$ for an unlabeled example with features $x^*$ by the rule
\[
y^* = \hat{\beta}_\lambda^T x^* + \hat{c}_\lambda
\]
Given training data
with labels $Y = (y_1,\hdots, y_n)$ and features $X = (x_1^T,\hdots,
x_n^T)$, the ridge regression 

\subsection{Principal Components Regression}

\section{Theory}

\subsection{Model}

We specify a generative model for data matrices $X = (x_1^T,\hdots,x_n^T)$ and $Y = (y_1, \hdots, y_n)$, which
together represent $n$ labeled examples.

Let $Z = (z_1^T,\hdots,z_n^T)$ be an $n \times r$ matrix of latent variables, where each
entry $z_{ij}$ is iid standard normal.  The latent variables are
related to $X$ and $Y$ in the following way.  Let
\[
X = Z\alpha + 1_n C^T + E
\]
where $\alpha$ is a $r \times p$ coefficient matrix with unit-norm
columns, $C$ is a fixed $p \times 1 $ intercept matrix, and $E$ is a
$n \times p$ random matrix of error terms which are iif $N(0,
\sigma_\epsilon^2)$.  Similarly, let
\[
Y = Z\gamma + c + \epsilon
\]
where $\gamma$ is a $r \times 1$ coefficient vector, $c$ is an
intercept term, and $\epsilon$ is a $n \times 1$ vector with entries
iid $N(0, \sigma^2_\epsilon)$.

The semi-supervised learning problem can be posed as follows.  Let
$\ell < n$ be the number of labeled examples, and let $Y_\ell$ be the
first $\ell$ rows of $Y$.  Supposing that only $X$ and $Y_\ell$ are
observed, and $\alpha$, $\gamma$, and $\sigma^2_\epsilon$ are unknown
parameters, the problem is to predict $\hat{Y}$ for all $n$ examples.
The squared-error loss is defined as
\[
||\hat{Y} - Y||^2
\]
and the goal is to choose a prediction method which minimizes the
\emph{risk}, or expected squared-error loss.

Of course, one can always set the first $\ell$ entries of $\hat{Y}$ to be
equal to $Y_\ell$, which guarantees zero error on those examples.  Thus the
challenge is to make a prediction for the unobserved entries of $Y$.

Throughout the paper we make the simplifying assumption that $r$, the
number of latent variables, is known; however, the problem of testing
the number of principal components $r$ is well-studied in the
statistics literature, and our analysis could be extended to
incorporate the case of unknown $r$.  Noting that the problems of
estimating the unknown intercept terms $C$ and $c$ as well as the
marginal variances of $X$ and $Y$ are also well-understood, we lose
little theoretical power and gain much clarity by making additional
assumptions that $C=0$, $c=0$, and that all the columns of $\alpha$
and $\gamma$ are unit-norm.  Note that as a result, $X$ and $Y$ have
zero marginal mean and equal marginal variances of $1 +
\sigma_\epsilon^2$, justifying the omission of the normalization step
normally employed in ridge regression.

\section{References}

\begin{itemize}
\item Zhu, X.  ``Semi-supervised learning literature survey.'' 2005.
\item Niyogi, P.  ``Manifold Regularization and semi-supervised learning:
Some theoretical analysis.''  2008.
\end{itemize}

\end{document}
