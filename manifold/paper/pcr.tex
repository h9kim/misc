\documentclass[11pt]{article} 

% packages with special commands
\usepackage{amssymb, amsmath}
\usepackage{epsfig}
\usepackage{array}
\usepackage{ifthen}
\usepackage{color}
\usepackage{fancyhdr}
\usepackage{graphicx}
%\usepackage{mathtools}
\definecolor{grey}{rgb}{0.5,0.5,0.5}

\begin{document}
\newcommand{\tr}{\text{tr}}
\newcommand{\E}{\textbf{E}}
\newcommand{\diag}{\text{diag}}
\newcommand{\argmax}{\text{argmax}}
\newcommand{\argmin}{\text{argmin}}
\newcommand{\Cov}{\text{Cov}}
\newcommand{\Vol}{\text{Vol}}
\pagestyle{fancy}

\title{Semi-supervised Principal Components Regression}

\author{Charles Zheng}

\maketitle

\begin{abstract}
\emph{Semi-supervised learning} refers to the problem of learning a
rule for predicting labels $y$ from features $x$, given training data
which includes both labeled examples $(x_1, y_1), \hdots, (x_\ell,
y_\ell)$ and unlabeled examples $x_{\ell + 1}, \hdots, x_n$.  A basic
question of this field is to characterize the conditions under which
the unlabeled examples can be used to improve generalization error.
We introduce a latent variable model for which semi-supervised
principal components regression is shown to outperform supervised
ridge regression.\\

Keywords: Semi-supervised, ridge regression,
principal components analysis, latent variables
\end{abstract}

\section{Introduction}

\subsection{Ridge Regression}

Ridge regression is a linear method for predicting a real-valued label
$y \in \mathbb{R}$ from a vector of real-valued features $x \in
\mathbb{R}^p$.  It is produces a regularized least squares estimate
for a coefficient vector $\hat{\beta}_\lambda$ and intercept
$\hat{c}_\lambda$, where $\lambda$ is a regularization parameter.  This
coefficient vector $\hat{\beta}_\lambda$ is used to predict the label
$y_*$ for an unlabeled example with features $x_*$ by the rule
\[
\hat{y}_\lambda = \hat{\beta}_\lambda^T x_* + \hat{c}_\lambda
\]

Given training data with labels $Y = (y_1,\hdots, y_n)$ and features
$X = (x_1^T,\hdots, x_n^T)$, and asssuming the feature matrix $X$ is
normalized to have columns with zero mean and unit variance,
the ridge regression coefficient vector is defined as
\[
\hat{\beta}_\lambda = (X^T X + \lambda I)^{-1} X^T Y_c
\]
and the intercept as
\[
\hat{c}_\lambda = \bar{y}
\]
where $Y_c = (y_1 - \bar{y},\hdots, y_n - \bar{y})$ and $\bar{y} =
\frac{1}{n}1^T Y$.

The dependence of the ridge-regression prediction
$\hat{y}_\lambda$ on the penalty $\lambda$ can be easily seen by
an application of the singular-value decomposition.
Suppose for now that $n < p$.
Write the singular-value decomposition of $X$ as
\[
\underbrace{X}_{n \times p} = \underbrace{U}_{n \times p}\underbrace{D}_{p \times p}\underbrace{V^T}_{p \times p}
\]
Then it is evident that
\[
\hat{\beta}_\lambda = V D (D^2 + \lambda)^{-1} U^T Y
\]
and therefore, writing $z_* = D^{-1} V^T x_*$, and $D = \diag(d_1,\hdots, d_p)$,
\[
\hat{y}_\lambda = z_*^T \diag\left(\frac{d_i^2}{d_i^2 + \lambda}\right) U^T Y + \bar{y}
\]
We see that increasing $\lambda$ tends to shrink the prediction
$\hat{y}$ towards $\bar{y}$.  Specifically, ridge regression shrinks
all of the principal directions $U^T Y$, applying more shrinkage to
the directions of lowest variance $d_i$ (Hastie 2008).

Shrinking $\hat{y}$ increases squared
bias (defined as $\E_{X, x_*, y_*}(\E_{Y}[\hat{y}] - y_*)^2$) but also reduces
variance (defined as $\E_{X, x_*, y_*}[(\hat{y} - \E_{Y}[\hat{y}])^2]$).  The
optimal $\lambda$ is determined by this bias-variance tradeoff, and in
practice, can be estimated in a data-depndent way, e.g. by using
cross-validation.

Due to the importance of choosing $\lambda$, it is usually not
practical to study the properties of ridge regression for fixed
$\lambda$.  One way to approach for studying ridge regression is
to let $f: X, Y \to \mathbb{R} $ be a selection rule for choosing
$\lambda$ in a data-dependent way, and define the ridge regression
prediction rule with selection rule $f$ by
\[
\hat{y}_f = \hat{\beta}_{f(X, Y)}^T x_* + \hat{c}_{f(X, Y)}
\]
Then the risk for the selection rule $f$ is
\[
R_f = \E||\hat{y}_f - y_*||^2
\]
However, most selection rules $f$ used in practice are difficult to
analyze.  Therefore, a more tractable approach is to define an \emph{oracle}-guided ridge regression procedure, which uses
\[
\lambda_{oracle} = \argmin_{\lambda} \E_{X, Y, x_*, y_*}[||y_* - \hat{y}_\lambda||^2]
\]
and hence incurs the risk
\[
R_{oracle} = \E||\hat{y}_{\lambda_{oracle}} - y_*||^2
\]
While it might appear that $R_{oracle}$ is obviously smaller than
$R_f$ for any selection rule $f$, since $R_{oracle}$ uses information
about the unknown join distribution of $(x,y)$, in fact there is no
existing proof that $R_{oracle}$ is necessarily smaller than $R_f$.
Nevertheless, it is widely believed that $R_{oracle}$ should be close
to $R_f$ for selection rules $f$ based on cross-validation.

\subsection{Principal Components Regression}

Principal components regression applies ordinary least-squares linear
regression to the top $k$ principal components of $X$.  Recalling the
singular-value decomposition $X = UDV^T$ from the previous section,
let $V_k$ denote the first $K$ columns of $V$, and let $T_k = XV_k$.
For a new point $x_*$, let $t_* = V_k^T x_*$ Define the $k$-principal
components regression coefficient vector
\[
\hat{\gamma}_{PCR-k} = (T_k^T T_k)^{-1} T_k^T Y
\]
and the prediction rule as
\[
\hat{y}_{PCR-k} = t_*^T \hat{\gamma}_{PCR-k}  + \hat{c}_{PCR-k}
\]
where if $X$ is centered,
\[
\hat{c} = \bar{y}
\]

It is easy to compare principal components regression with ridge
regression by writing $\hat{y}_{PCR-k}$ in terms of the SVD of $X$,
\[
\hat{y}_{PCR-k} = z_*^T \diag(1_k, 0) U^T Y + \bar{y}
\]
where as before $z^* = V x_*$.  While ridge regression shrinks the
principal directions $U^T Y$ depending on the variance of the
corresponding principal components, principal components regression
``kills'' all but the top $k$ principal directions without shrinking
the remaining $k$.  Hence, in a supervised setting PCR and ridge
regression have very similar behavior.  However, we will see that
unlike with ridge regression, PCR has the potential to improve with
the addition of unlabeled examples.

\section{Theory}

\subsection{Model}

We specify a generative model for data matrices $X = (x_1^T,\hdots,x_n^T)$ and $Y = (y_1, \hdots, y_n)$, which
together represent $n$ labeled examples.

Let $Z = (z_1^T,\hdots,z_n^T)$ be an $n \times r$ matrix of latent variables, where each
entry $z_{ij}$ is iid standard normal.  The latent variables are
related to $X$ and $Y$ in the following way.  Let
\[
X = Z\alpha + 1_n C^T + E
\]
where $\alpha$ is a $r \times p$ coefficient matrix with unit-norm
columns, $C$ is a fixed $p \times 1 $ intercept matrix, and $E$ is a
$n \times p$ random matrix of error terms which are iif $N(0,
\sigma_\epsilon^2)$.  Similarly, let
\[
Y = Z\gamma + c + \epsilon
\]
where $\gamma$ is a $r \times 1$ coefficient vector, $c$ is an
intercept term, and $\epsilon$ is a $n \times 1$ vector with entries
iid $N(0, \sigma^2_\epsilon)$.

The semi-supervised learning problem can be posed as follows.  Let
$\ell < n$ be the number of labeled examples, and let $Y_\ell$ be the
first $\ell$ rows of $Y$.  Supposing that only $X$ and $Y_\ell$ are
observed, and $\alpha$, $\gamma$, and $\sigma^2_\epsilon$ are unknown
parameters, the problem is to predict $\hat{Y}$ for all $n$ examples.
The squared-error loss is defined as
\[
||\hat{Y} - Y||^2
\]
and the goal is to choose a prediction method which minimizes the
\emph{risk}, or expected squared-error loss.

Of course, one can always set the first $\ell$ entries of $\hat{Y}$ to be
equal to $Y_\ell$, which guarantees zero error on those examples.  Thus the
challenge is to make a prediction for the unobserved entries of $Y$.

Throughout the paper we make the simplifying assumption that $r$, the
number of latent variables, is known; however, the problem of testing
the number of principal components $r$ is well-studied in the
statistics literature, and our analysis could be extended to
incorporate the case of unknown $r$.  Noting that the problems of
estimating the unknown intercept terms $C$ and $c$ as well as the
marginal variances of $X$ and $Y$ are also well-understood, we lose
little theoretical power and gain much clarity by making additional
assumptions that $C=0$, $c=0$, and that all the columns of $\alpha$
and $\gamma$ are unit-norm.  Note that as a result, $X$ and $Y$ have
zero marginal mean and equal marginal variances of $1 +
\sigma_\epsilon^2$, justifying the omission of the normalization step
normally employed in ridge regression.

\subsection{Ridge Regression}

Write the following singular value decompositions
\[
\underbrace{\alpha}_{r \times p} = \underbrace{\eta}_{r \times p} \underbrace{\tilde{D}}_{p \times p} \underbrace{\tilde{V}^T}_{p \times p}
\]
\[
\underbrace{X}_{r \times p} =\sqrt{n} \underbrace{U}_{n \times p}\underbrace{D}_{p \times p}\underbrace{V^T}_{p \times p}
\]
The scaling of $\sqrt{n}$ is so that $D$ is $O(1)$.
Also rescale the ridge regression penalty so that
\[
\hat{\beta}_\lambda = (X^T X + n \lambda I)^{-1} X^T Y
\]
Write
\[
Z^TZ = n (I + \delta_Z)
\]
\[
\tilde{V}^T V  = I +  \delta_V
\]
Then
\begin{align*}
\hat{y}_\lambda - y_* =&
x_*^T (X^T X + n \lambda I)^{-1} X^T Y - y_*\\
=& n^{-1} (z_*^T \alpha + E_*) V ( D^2 +\lambda I)^{-1} V^T (\alpha^T Z^T Z \gamma + E^T Z\gamma + \alpha^T Z^T \epsilon + E^T \epsilon) - z_*^T \gamma - \epsilon
\\=& z_*^T (\eta \tilde{D}^2 (D^2 + \lambda)^{-1} \eta^T \gamma - \gamma)
\\&+ z_*^T(\eta \tilde{D} \delta_v (D^2 + \lambda)^{-1} (I + \delta_V^T) \tilde{D}^T \eta^T (I + \delta_Z) \gamma)
\\&+ z_*^T(\eta \tilde{D} (I + \delta_V) (D^2 + \lambda)^{-1} \delta_V^T \tilde{D}^T \eta^T (I + \delta_Z) \gamma)
\\&+ z_*^T(\eta \tilde{D} (I + \delta_V) (D^2 + \lambda)^{-1} (I + \delta_V^T) \tilde{D}^T \eta^T \delta_Z \gamma)
\\&+ E_* V (D^2 +\lambda I)^{-1} V^T \alpha^T (I + \delta_Z) \gamma
\\&+ n^{-1} x_*^T V (D^2 +\lambda I)^{-1} V^T (E^T Z\gamma + \alpha^T Z^T \epsilon + E^T \epsilon)  - \epsilon
\end{align*}

\section{References}

\begin{itemize}
\item Hastie, Tibshirani, Friedman.  ``The Elements of Stastistical Learning,'' 2008.
\item Zhu, X.  ``Semi-supervised learning literature survey.'' 2005.
\item Niyogi, P.  ``Manifold Regularization and semi-supervised learning:
Some theoretical analysis.''  2008.
\end{itemize}

\end{document}
