\documentclass[11pt]{article} 

% packages with special commands
\usepackage{amssymb, amsmath}
\usepackage{epsfig}
\usepackage{array}
\usepackage{ifthen}
\usepackage{color}
\usepackage{fancyhdr}
\usepackage{graphicx}
%\usepackage{mathtools}
\definecolor{grey}{rgb}{0.5,0.5,0.5}

\begin{document}
\newcommand{\tr}{\text{tr}}
\newcommand{\E}{\textbf{E}}
\newcommand{\diag}{\text{diag}}
\newcommand{\argmax}{\text{argmax}}
\newcommand{\argmin}{\text{argmin}}
\newcommand{\Cov}{\text{Cov}}
\newcommand{\Vol}{\text{Vol}}
\pagestyle{fancy}

\title{Semi-supervised learning via matrix completion}

\author{Charles Zheng and Trevor Hastie}

\maketitle

\begin{abstract}
\emph{Matrix completion} refers to the problem of inferring the
missing entries of a given $m\times n$ matrix $R$.  As such, the
problem of \emph{semi-supervised prediction} can be interpreted as a
special case of matrix completion.  The setting of semi-supervised
learning involves an $m \times p$ matrix of features and a partially
observed $m \times 1$ vector of labels; the goal is to derive a
prediction rule for predicting the unobserved labels, as well as
labels for newly observed features.  Matrix completion can be applied
in this setting to complete the combined matrix of features and
labels, $R = (X Y)$; here, the only missing entries are the unobserved
labels.  Hence, it is conceivable that existing approaches for matrix
completion, such Soft-Impute (Mazumder et al 2010) could be applied to
the problem of semi-supervised learning.  However, generalizing to new
examples requires the additional step of interpreting the result of
the matrix completion as a \emph{prediction rule}.  Here we derive a
prediction rule for Soft-Impute, and examine its qualities for the
semi-supervised prediction problem.
\end{abstract}

\section{Introduction}

Suppose $R_{m \times n}$ is a partially observed matrix: we only
observe $R_{ij}$ for $(i, j) \in \Omega$.  Let $B$ denote a $m \times
n$ boolean matrix where $B_{ij} = I{(i, j) \in \Omega}$; we will write
$R \circ B$ to denote the matrix of observed entries.

Here is one way to predict the missing entries.  Consider the
following matrix completion problem (Mazumder et al 2010).
$$
\text{minimize}_Z \frac{1}{2}\sum_{(i,j) \in \Omega} (R_{ij} - Z_{ij})^2 + \lambda ||Z||_\star 
$$ After solving this optimization problem, use $Z_{ij}$ as a
prediction of the missing entries $R_{ij}$.

Consider the solution $Z(\lambda)$ of this optimization problem.  If $r =
rank(Z(\lambda))$, then also $Z(\lambda) = U(\lambda)V(\lambda)^T$, where $U(\lambda)$, $V(\lambda)$ are the solutions to
$$
\text{minimize}_{U_{m\times r}, V_{n \times r}} \frac{1}{2} \sum_{(i, j) \in \Omega} (R_{ij} - (UV^T)_{ij})^2 + \frac{\lambda}{2}(||U||^2_F + ||V||^2_F)
$$

\section{Semi-supervised learning}

Can we interpret matrix completion as a form of semi-supervised
learning?  In semi-supervised learning, we have observed covariates
$x^1, ... , x^m \in \mathbb{R}^p$ and \emph{partially} observed responses
$y^1, ..., y^{m_0}$ where $m_0 < m$.  Let $X_{m \times p}$ denote the
matrix of covariates and $Y = (y^1, ... , y^m)$ denote the full set of
observed and unobserved responses.

A \emph{supervised} approach would be to fit a model to the fully
observed pairs,
$$ y^i \approx x^i \beta + \beta_0 $$
for $i = 1,..., m_0$,
and then predict the unobserved responses as
$\hat{y}^i = x^i \beta + \beta_0$ for $i = m_0 + 1, ... , m$.

Now consider using matrix completion to solve the problem.  Define the
matrix $$R_{m \times n} = [X | Y]$$ where $n = p+1$.  Here we have
observed $R_{ij}$ for all $j = 1, ... , n-1$ and for all $j = n$, $i =
1, ..., m_0$.  The problem of predicting $y^{m_0 + 1},..., y^m$ is
equivalent to predicting the missing elements $R^{m_0 + 1, n}, ...,
R^{m, n}$.  This approach is \emph{semi-supervised} because it uses
information from all the covariate vectors $x^1,\hdots, x^m$, not just
the covariates with observed responses.

This suggests thinking of matrix completion as a method for learning a
predictive model, which can be used to predict $Y$ given $X$.
However, it is not perfectly straightforward to interpret matrix
completion as a predictive model like regression.  In regression, the
model gives a \emph{prediction rule} for labelling a new observation
$X^*$.  In matrix completion, if we wanted to predict $Y^*$ for a new
observation $X^*$, we could do so by extending the matrix $R$ by one
row and re-running matrix completion.  However, we argue that this
process cannot be described as a ``prediction rule" since it involves
retraining the entrie model. In the following, we demonstrate that
there \emph{is} a way to interpret matrix completion in terms of a
prediction rule: our proposed rule gives different results than
re-running matrix completion on an extended matrix $R$.

\section{Prediction rules for matrix completion}

The problem of matrix completion can be phrased in terms of a
population-based model as follows.  Suppose we have a real-valued
random vector $R \in \mathbb{R}^n$ with some unknown distribution $R
\sim F$.  Meanwhile, there also exists an $n$-dimensional boolean
random vector $B$, such that $R$ and $B$ have a joint distribution
$G$; alternatively, say that $B$ has a distribution $G_R$ conditional
on $R$.  Let $(R^1, B^1), ... , (R^m, B^m)$ be a sample of iid
realizations from the joint distribution $G$.  We then observe
$R_{ij}$ for each $(i, j) \in [m] \times [n]$ where $B_{ij} = 1$.

This yields our training set, from which we can derive a \emph{prediction
rule} for inferring missing entries of a new observation $R^*$ for
which we only observe the entries determined by $B^*$; more formally,
a function $f$ which maps a partially observed vector $(r_{* i_1},...,
r_{* i_k})$ to a prediction of both observed and unboserved entries
$(z_1, ..., z_n)$.  Here we do not require the predictions to match on
observed and unobserved entries.  For notational purposes let $R^i
\circ B^i$ denote the observed entries of row $i$, hence $f$ maps $R^i
\circ B^i$ to a prediction $Z^i \in \mathbb{R}^n$.

One recognizes that the goal of matrix completion is to find a
prediction rule which minimizes the squared error of the missing
entries within the sample:

$$
\text{pre. error} = \sum_{i = 1}^m \sum_{j: B_{ij} = 0} (R_{ij} - f(R^i)_{j})^2
$$

However, it is not immediately obvious as to whether there always
exists such a function $f$ which describes the prediction made by
matrix completion: i.e. $f$ satisfying $f(R^i \circ B^i) =
Z(\lambda)^i$, where $Z(\lambda)_{m \times n}$ is a minimizer of the
objective

$$
\text{minimize}_Z \frac{1}{2}\sum_{(i,j) \in \Omega} (R_{ij} - Z_{ij})^2 + \lambda ||Z||_\star 
$$

In fact, it is easy to see that such a function $f$ always exists:
just pick any function which maps $R^i \circ B^i$ to $Z(\lambda)^i$,
and takes an arbitrary value anywhere else.  The function is
well-defined, because we can show that there exists $Z(\lambda)$ where
for any $i, j \in [m]$ such that $R^i\circ B^i = R^j \circ B^j$, we
have $Z(\lambda)^i = Z(\lambda)^j$.

However, now the problem is that the function $f$ is not unique, and
it also hardly resembles a proper \emph{prediction rule} in the sense
that rather than summarizing the data, it actually requires more
information (order $mn$) to describe than the original data.  More
importantly, such an arbitrarily constructed *prediction rule* can
hardly be expected to generalized to new examples.  To elaborate,
suppose that we consider the goal of minimizing the generalization
error on new examples, defined as:

$$
\text{gen. error} = \text{E}_{R^*, B^*} \sum_{j: B_{*j} = 0} (R_{* j} - f(R^*)_j)^2
$$

We resolve all three of these issues by presenting a prediction rule
which is uniquely defined, which can be compactly described, and which
(we will show) generalizes well under the given assumptions.  To
derive the prediction rule $f$, recall that $Z(\lambda)$ can be
written as $Z(\lambda) = U(\lambda)V(\lambda)^T$, where $U(\lambda)$
and $V(\lambda)$ are the solution to
$$
\text{minimize}_{U_{m\times r}, V_{n \times r}} \frac{1}{2} \sum_{(i, j) \in \Omega} (R_{ij} - (UV^T)_{ij})^2 + \frac{\lambda}{2}(||U||^2_F + ||V||^2_F)
$$
for $r = Rank(Z(\lambda))$.

Now the key observation: supposing we only knew $V(\lambda)$, we could
recover each row of $U(\lambda)$ using only information from the
corresponding row of $R \circ B$.  To see this, rewriting the
objective function having fixed $V$ (so we are only minimizing over
$U$) and in terms of the individual rows, we get
$$
U(\lambda) = \text{argmin}_{U_{m\times r}} \frac{1}{2} \sum_{i \in [m]} \sum_{j : (i,j) \in \Omega} (R_{ij}  - (U^i V(\lambda)^T)_j)^2 + \frac{\lambda}{2}||U^i||^2
$$
hence the objective function separates over rows of $U$, and
$$
U(\lambda)^i = \text{argmin}_{\mathbb{R}^r} \sum_{j : (i,j) \in \Omega} (R_{ij}  - (U^i V(\lambda)^T)_j)^2 + \lambda||U^i||^2
$$
But this is simply a least squares problem. For fixed $i$, let $j_1,..., j_{k_i}$ be the indices $j$ such that $B_{ij} = 1$.
Let $R^{[B^i]}$ denote a column vector consisting only of the observed entries of $R^i$, i.e. $R^{[B^i]} = (R_{ij_1}, ... , R_{ij_{k_i}})$.  
Meanwhile, let $V^{[B^i]}$ denote the $k_i \times r$ submatrix of $V(\lambda)$ with rows
$V^{j_1} , ...,  V^{j_{k_i}}$.
Then we have
$$
(U(\lambda)^i)^T = ((V^{[B^i]})^T V^{[B^i]} + \lambda I_r)^{-1} (V^{[B^i]})^T R^{[B^i]}
$$ So far we have described a way of getting $U(\lambda)^i$ from $R^i
\circ B^i$.  In order to specify the prediction rule $f$, it remains
to use the fact that $Z^i = U(\lambda)^i V(\lambda)^T$.

Hence our prediction rule is:
\[
f(R^* \circ B^*) = [((V^{[B^*]})^T V^{[B^*]} + \lambda I_r)^{-1} (V^{[B^*]})^T R^{[B^*]}]^T V(\lambda)^T
\]

This prediction rule is very general in the sense that it can take input with any missingness pattern.  For the problem of semi-supervised learning, we can specialize to the missingness pattern where only the last entry of $R^*$, corresponding the to response, is missing.  This yields a prediction rule for semi-supervised learning
\[
y^* = \gamma x^*
\]
where
\[
\gamma = V(\lambda)^n ((V^{(-n)})^T V^{(-n)} + \lambda I_r)^{-1} (V^{(-n)})^T
\]

Hence, the prediction rule for matrix completion is a special form of linear regression!

\end{document}
