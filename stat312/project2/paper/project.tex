%%%%%%%%%%%%%%%%%%%%%%%%%%%%%%%%%%%%%%%%%
% Beamer Presentation
% LaTeX Template
% Version 1.0 (10/11/12)
%
% This template has been downloaded from:
% http://www.LaTeXTemplates.com
%
% License:
% CC BY-NC-SA 3.0 (http://creativecommons.org/licenses/by-nc-sa/3.0/)
%
%%%%%%%%%%%%%%%%%%%%%%%%%%%%%%%%%%%%%%%%%

%----------------------------------------------------------------------------------------
%	PACKAGES AND THEMES
%----------------------------------------------------------------------------------------

\documentclass{beamer}

\mode<presentation> {

% The Beamer class comes with a number of default slide themes
% which change the colors and layouts of slides. Below this is a list
% of all the themes, uncomment each in turn to see what they look like.

%\usetheme{default}
%\usetheme{AnnArbor}
%\usetheme{Antibes}
%\usetheme{Bergen}
%\usetheme{Berkeley}
%\usetheme{Berlin}
%\usetheme{Boadilla}
%\usetheme{CambridgeUS}
%\usetheme{Copenhagen}
%\usetheme{Darmstadt}
%\usetheme{Dresden}
%\usetheme{Frankfurt}
%\usetheme{Goettingen}
%\usetheme{Hannover}
%\usetheme{Ilmenau}
%\usetheme{JuanLesPins}
%\usetheme{Luebeck}
\usetheme{Madrid}
%\usetheme{Malmoe}
%\usetheme{Marburg}
%\usetheme{Montpellier}
%\usetheme{PaloAlto}
%\usetheme{Pittsburgh}
%\usetheme{Rochester}
%\usetheme{Singapore}
%\usetheme{Szeged}
%\usetheme{Warsaw}

% As well as themes, the Beamer class has a number of color themes
% for any slide theme. Uncomment each of these in turn to see how it
% changes the colors of your current slide theme.

%\usecolortheme{albatross}
%\usecolortheme{beaver}
%\usecolortheme{beetle}
%\usecolortheme{crane}
%\usecolortheme{dolphin}
%\usecolortheme{dove}
%\usecolortheme{fly}
%\usecolortheme{lily}
%\usecolortheme{orchid}
%\usecolortheme{rose}
%\usecolortheme{seagull}
%\usecolortheme{seahorse}
%\usecolortheme{whale}
%\usecolortheme{wolverine}

%\setbeamertemplate{footline} % To remove the footer line in all slides uncomment this line
%\setbeamertemplate{footline}[page number] % To replace the footer line in all slides with a simple slide count uncomment this line

%\setbeamertemplate{navigation symbols}{} % To remove the navigation symbols from the bottom of all slides uncomment this line
}

\usepackage{graphicx} % Allows including images
\usepackage{booktabs} % Allows the use of \toprule, \midrule and \bottomrule in tables
\usepackage{multirow}
%----------------------------------------------------------------------------------------
%	TITLE PAGE
%----------------------------------------------------------------------------------------


\title[Experimental Design Project]{Uniform Image Selection for fMRI studies}

\author{Charles Zheng} % Your name
\institute[Stanford] % Your institution as it will appear on the bottom of every slide, may be shorthand to save space
{Stanford University}
\date{\today} % Date, can be changed to a custom date

\begin{document}

\begin{frame}
\titlepage % Print the title page as the first slide
\end{frame}

\section{Introduction}

\begin{frame}
\frametitle{How should researchers choose images?}
\begin{itemize}
\item \emph{Relevance}. Using natural images, because we care about how the brain percieves typical scenes in the real world.
\item \emph{Control}. Using artificially generated images (like gratings) with known parameters
\item \emph{Diversity}. Using a broad set of images to convincingly demonstrate generalizability
\end{itemize}
In this work we only deal with the question of \emph{diversity}
\end{frame}

\begin{frame}
\frametitle{Questions about diversity}
\begin{enumerate}
\item How can we define diversity?
\item How can researchers maximize this measure of diversity in their image set?
\item How does diversity affect metrics such as classification performance?
\item How can diversity improve generalizability?
\end{enumerate}

To explore these questions, we use the dataset from \emph{Kay et al.}
\end{frame}

\section{Defining diversity}

\frame{\sectionpage}

\begin{frame}
\frametitle{Defining diversity}
\begin{itemize}
\item A first step in defining diversity is \emph{dimensionality reduction} of images to a low-dimensional representation
\item Next, we need a \emph{distance metric} on the low-dimensional space
\item With coordinates and a distance metric, we can define a measure of diversity
\end{itemize}
\end{frame}

\begin{frame}
\frametitle{Dimensionality reduction}
\begin{itemize}
\item Suppose we have ``training data'' for $K$ images
\item A $K \times p_f$ matrix $X$ of image features (e.g. Gabor filters)
\item A $K \times p_v$ matrix $Y$ of mean voxel responses
\item Use \emph{sparse canonical correlation analysis} to find a $p_f
  \times d$ image basis $U$ and a $p_v \times d$ voxel basis $V$
\item Use the image basis $U$ to define coordinates
\end{itemize}
\end{frame}

% demonstration in data

\begin{frame}
\frametitle{A measure of diversity}
\begin{itemize}
\item We have coordinates $z$ for any image, and a distance metric $d(z_1, z_2)$
\item Suppose that the coordinates of all images form a continuous, compact support set $S$
\item We will define the diversity of a distribution $P$ supported on $S$
\item Let $Z_1$, $Z_2$ be random points drawn from $P$
\item Let $\kappa(\cdot)$ be a bounded monotone decreasing function which goes to zero at infinity
\item Define the diversity of $P$ as
\[
-\mathbb{E}[\kappa(d(Z_1, Z_2))]
\]
\end{itemize}
\end{frame}

\begin{frame}
\frametitle{A measure of diversity}
\begin{itemize}
\item Let $\kappa(\cdot)$ be a bounded monotone decreasing function which goes to zero at infinity
\item Define the diversity of $P$ as
\[
-\mathbb{E}[\kappa(d(Z_1, Z_2))]
\]
\item We did not define the function $\kappa$... but in some sense the choice is unimportant!
\item Claim: let $h$ be a bandwidth, and let 
\[P_{\kappa, h} = \text{argmin} \mathbb{E}[k(d(Z_1, Z_2)/h)]\]
Then let $P_\kappa = \lim_{h \to 0} P_{\kappa, h}$.
There exist a unique distribution $U$ such that for all $\kappa$ satisfying a \emph{postive definite} condition, $P_\kappa = U$.
We call $U$ the \emph{uniform distribution}
\end{itemize}
\end{frame}

% measuring diversity of training and test set

\section{Maximizing diversity}

\frame{\sectionpage}



\end{document}












