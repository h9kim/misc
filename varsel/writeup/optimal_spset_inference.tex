\title{Inference for the optimal sparse prediction set}
\author{Charles Zheng and Trevor Hastie}
\date{\today}

\documentclass[12pt]{article} 

% packages with special commands
\usepackage{amssymb, amsmath}
\usepackage{epsfig}
\usepackage{array}
\usepackage{ifthen}
\usepackage{color}
\usepackage{fancyhdr}
\usepackage{graphicx}
\usepackage{mathtools}
\usepackage{csquotes}
\definecolor{grey}{rgb}{0.5,0.5,0.5}

\begin{document}
\maketitle

\newcommand{\tr}{\text{tr}}
\newcommand{\E}{\textbf{E}}
\newcommand{\diag}{\text{diag}}
\newcommand{\argmax}{\text{argmax}}
\newcommand{\Cov}{\text{Cov}}
\newcommand{\Var}{\text{Var}}
\newcommand{\argmin}{\text{argmin}}
\newcommand{\Vol}{\text{Vol}}
\newcommand{\comm}[1]{}

\newcommand{\bx}{\boldsymbol{x}}
\newcommand{\by}{\boldsymbol{y}}
\newcommand{\bX}{\boldsymbol{X}}
\newcommand{\bY}{\boldsymbol{Y}}


\section{Introduction}

\subsection{Linear prediction with fixed design, saturated normal model}

Let $X$ be a fixed $n \times p$ design matrix, and suppose $Y \sim
N(\mu, \sigma^2 I)$.  Given observed $X$ and $Y$, we wish to predict
$Y^*$, an unobserved, independent draw from $N(\mu, \sigma^2 I)$.
Under the \emph{linear prediction problem}, the prediction $\hat{Y}$
takes the form $\hat{Y} = X\gamma$ for some coefficient vector
$\hat{\gamma}$ to be determined in a data-dependent way.  Our objective is
to choose $\hat{\gamma}$ in order to minimize the prediction risk,
$$
R(\gamma) = \E||Y^* - X\hat{\gamma}||^2.
$$

\subsection{Classical model selection}

A classical approach to linear regression is to employ \emph{model
selection} to first select a subset $S \subset \{1,..,p\}$ of the
covariates; the selected \emph{model} $M_S$ is the set of all vectors
$$
M_S = \gamma\in \mathbb{R}^p\text{ such that }\gamma_i = 0\text{ for all }i \notin S.
$$  
If we define $\gamma_S$ to be the best coefficient vector in $M_S$, i.e.
$$\gamma_S^* = \argmin_{\gamma \in M_S} R(\gamma)$$,
then the nonzero entries of $\gamma_S^*$ are given by $\beta_S$,
$$
\beta_S = (X_S^T X_S)^{-1} X_S^T \mu
$$
where $X_S$ is the submatrix of $X$ obtained by taking the columns in $S$.
Unconditionally, there exists an unbiased estimator of $\beta_S$,
$$
\hat{\beta}_S = (X_S^T X_S)^{-1} X_S^T Y.
$$
(Though if $Y$ is used to select $S$, $\hat{\beta}_S$ is no longer
unbiased conditional on the selected $S$.)  In any case, we take
$\hat{\gamma}$ to be the vector with $\hat{\gamma}_i = 0$ for all
$i \notin S$, and with nonzero entries given by $\hat{\beta}_S$.

Numerous methods exist for \emph{model selection}. Classical model
selection techniques combine a search method with a model selection
criterion: search methods include best-subset, forward stepwise, and
backwards stepwise; while the model selection criteria include AIC and
BIC.

\subsection{Constrained optimal model}

Let $\mathcal{S}$ be some family of subsets of $\{1,\hdots, p\}$;
for instance, the set of \emph{sparse models}
$$
\mathcal{S} = \{S \subset\{1,\hdots, p\}: |S| \leq k\}
$$
for some integer $k$.

Define the oracle risk of a subset $S$ as $\min_{\gamma \in M_S} R(\gamma)$.
That is the risk we would achieve if we knew the distribution of $Y$, but were constrained to the model $M_S$.
Define the constrained optimal model as the set in $\mathcal{S}$ with the least oracle risk:
\[
S^* = \argmin_{S \in \mathcal{S}} \min_{\gamma \in M_S} R(\gamma)
\]

Some simple algebra shows that
\[
\min_{\gamma \in M_S} R(\gamma) = n\sigma^2 + ||\mu||^2 - \mu^T X_S (X_S^T X_S)^{-1} X_S \mu.
\]

Therefore, defining the quantity $V_S$ (``variance explained'')
$$
V_S = \mu^T X_S (X_S^T X_S)^{-1} X_S \mu,
$$
we see that an equivalent definition of the constrained optimal model
is the set in $\mathcal{S}$ which maximizes the variance explained,
$$
S^* = \argmax_{S \in \mathcal{S}} V_S.
$$

In applications, one can imagine several motivations for attempting to recover an optimal constrained model.
\begin{itemize}
\item To find a parsimonious set of predictors, $S^*$.
\item To achieve good prediction error in cases where at least one good sparse linear model exists.
\item To infer the causal parents of $Y$, under the assumption that the data is generated by a sparse linear model, $\mu = X_{S^\dagger} \beta_{S^\dagger}$, where $X_{S^\dagger}$ are the causal parents of $Y$.
Then the optimal sparse prediction set $S^*$ coincides with
$S^\dagger$ as long as one selects the correct sparsity $k =
|S^\dagger|$.  If $k > |S^\dagger|$, then $S^*$ is non-unique, but all
optimal prediction sets are supersets of $S^\dagger$.
\end{itemize}

\subsection{Inferring the constrained optimal model}

Having defined $S^*$, we now consider the problem of inference.
A \emph{point estimate} for $S^*$ would be a single model
$S \in \mathcal{S}$, such that $S$ is ``close'' to $S^*$ according to some metric.
Popular criteria include any combination of:
\begin{itemize}
\item Minimizing prediction \emph{regret}, $\E[V_{S^*}]-V_S$.
\item Controlling \emph{familywise error}, $\Pr[|S \setminus S^*| > 1]$.
\item Controlling \emph{false discovery proportion} (FDP), $|S \setminus S^*|/|S^*|$.
\item Maximizing true discoveries, $\E[|S \cap S^*|]$.
\end{itemize}
For instance, one may try to minimize prediction regret while
simultaneously controlling false discovery rate (expected FDP).
Or one could also consider prediction regret as the only criterion.

In this work, we do not consider the point estimation problem.  Rather, we consider the \emph{set estimation} problem,
that is, construct a set $\mathcal{A}$ of candidate models $\mathcal{A} \subset \mathcal{S}$,
such that we ensure that the type I error probability,
\[
p_e = \sup_{\mu \in \mathbb{R}^p} \Pr[S^* \notin \mathcal{A}]
\]
is kept low, over all possible values of the unknown parameter $\mu \in \mathbb{R}^p$.
For the time being, we assume that the noise level $\sigma^2$ is known.

Classical frequentist error control requires that the procedure
satisfies $p_e \leq \alpha$ for some level $\alpha \in [0,1]$.  In
such a case, $\mathcal{A}$ is considered a \emph{valid} confidence
set.  From a practical standpoint, a useful confidence set would not
only be valid (or approximately valid) but also be small in size.
After all, taking $\mathcal{A} = \mathcal{S}$ gives a trivially valid
confidence set, but this is of no use.

Alternatively, a less stringent condition is to require \emph{consistency} in some asymptotic regime:
that is, for a sequence of problems with parameters $X^{(i)}, \mu^{(i)}$ for $i = 1,\hdots$, we have
\[
\lim_{i \to \infty} p_e^{(i)} \to 1
\]
where $p_e^{(i)}$ is the type I error for the $i$th problem in the
sequence.  Again, after we establish a proposed procedure
as \emph{consistent}, we would further hope to show (perhaps
empirically) that has additional practical properties such as produce
small confidence sets and have good finite-sample performance.

[Where has this problem been studied?]

\section{A first attempt}

A first attempt to the problem follows from the following idea.  First of all, define the estimated variance explained as
\[
\hat{V}_S = Y^T X_S (X_S X_S)^{-1} X_S Y - \sigma^2 |S|
\]
We subtract $\sigma^2 |S|$ in order to apply Mallow's $C_p$ correction.

Write $Y = \mu + \epsilon$, where $\epsilon \sim N(0, \sigma^2 I)$.  Further define $\delta = X^T \mu$ and $z = X^T \epsilon$.  Then
\[
\hat{V}_S = V_S + 2\delta_S^T (X_S^T X_S)^{-1}z_S + z_S^T (X_S^T X_S)^{-1} z_S - \sigma^2 |S|.
\]

We have
\[
\E[\hat{V}_S] = V_S + \E[\tr[z_S^T (X_S^T X_S)^{-1} z_S]] -\sigma^2 |S| = V_S + \E[(X_S^T X_S)^{-1} (\sigma^2 X_S^T X_S)] -\sigma^2 |S|= V_S.
\]

Under appropriate conditions, we can ensure that $\hat{V}_S$ have an
asymptotically jointly multivariate normal distribution for
$S \in \mathcal{S}.$  For instance, consider a sequence of problems where
\begin{enumerate}
\item $p$ and $\sigma^2$ remains fixed while $n$ grows.
\item $X^{(i)T} X^{(i)}/n$ converges to some matrix $\Sigma$.
\item The sequence $\{\mu^{(i)}\}_{i=1}^\infty$ is chosen so that the ``difficulty'' of the problem remains constant as $i \to \infty$.
\end{enumerate}
The first two conditions are straightforward enough, but



\end{document}



