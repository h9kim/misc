\documentclass[11pt]{article} 

% packages with special commands
\usepackage{amssymb, amsmath}
\usepackage{epsfig}
\usepackage{array}
\usepackage{ifthen}
\usepackage{color}
\usepackage{fancyhdr}
\usepackage{graphicx}
\usepackage{mathtools}
\usepackage{amsthm}
\definecolor{grey}{rgb}{0.5,0.5,0.5}

\begin{document}
\newcommand{\tr}{\text{tr}}
\newcommand{\E}{\textbf{E}}
\newcommand{\diag}{\text{diag}}
\newcommand{\argmax}{\text{argmax}}
\newcommand{\Cov}{\text{Cov}}
\newcommand{\Var}{\text{Var}}
\newtheorem{theorem}{Theorem}

Charles Zheng

\section{Background}

\begin{theorem}
\label{bound}
Define $\phi(x) = \frac{1}{\sqrt{2\pi}}e^{-x^2/2}$ and
\[
\Psi(x) = \int_x^\infty \phi(t) dt
\]
for $x \in \mathbb{R}$.
Then
\[
\left(\frac{1}{x^3}-\frac{1}{x}\right) \phi(x) \leq \Psi(x) \leq \frac{1}{x}\phi(x)
\]
\end{theorem}

\begin{theorem}
\label{bt}
\textbf{(Borell-TIS inequality)}
Let $f_t$ be a gaussian process such that $\mathbb{E}[f_t] = 0$
Then on any measurable set $D$, and $u > 0$, 
\[
\mathbb{P}[\sup_D f_t > u + \mathbb{E}[\sup_D f_t]] \leq \exp(-u^2/(2\sigma_{max}^2))
\]
where
\[
\sigma_{max} = \sup_D \E[f_t^2]
\]
\end{theorem}

\begin{theorem}\label{slepian}
\textbf{(Slepian's inequality)}
If $f$ and $g$ are as bounded, centered gaussian processes, and
\[
\mathbb{E}[(f_t-f_s)^2] \leq \mathbb{E}[(g_t-g_s)^2]
\]
then
\[
\mathbb{P}[\sup_{t \in D} f_t  > u] \leq \mathbb{P}[\sup_{t \in D} g_t  > u]
\]
\end{theorem}

\section{Supremum of an isotropic GP}

\begin{theorem}
Let $f_t$ be a gaussian process on $\mathbb{R}$, with $\Cov(f_t,f_u)
= C(t-u)$, where $C(0) = 1$, and $C(t) \to 0$ as $||t|| \to \infty$.
Then supposing $f_t$ is bounded on $[0,1]$,
\[
\mathbb{P}\left(\lim_{T \to \infty}\frac{\sup_{[0,T]}
    f_t}{\sqrt{2 \log(T)}} = 1
\right) = 1
\]
\end{theorem}

\noindent\textbf{Proof.}

It suffices to prove
\[
\mathbb{P}\left( 1-\varepsilon \leq \lim_{T \to \infty}\frac{\sup_{[0,T]}
    f_t}{\sqrt{2 \log(T)}} \leq 1+\varepsilon
\right) = 1
\]
for arbitrary $\epsilon \in (0,1)$.

Take $\epsilon \in (0,1)$.

Find $\tau > 0$ such that $C(\tau) <
\frac{\varepsilon}{2-\varepsilon}$ and find $T_0 > 0$ such that $T > \max\{ 2\tau,\frac{2-\varepsilon}{\varepsilon} \log(2\tau), e^{\frac{1-C(\tau)}{(1-\varepsilon)^2}}\}$.  For each of $n =
1,\hdots,$, let $T = T_0 + n - 1$, and let let $m = \lfloor
\frac{T+1}{\tau} \rfloor$.  Define $t_k = k\tau$ for $k = 1,\hdots,m$.
Let $Z_1,\hdots, Z_m$ be iid $N(0,1-C(\tau))$.  We have
\[
\mathbb{E}[(Z_i - Z_j)^2] \leq 2(1-C(\tau)) \leq 2(1-C((i-j)\tau)) = \mathbb{E}[(f_{t_i}-f_{t_j})^2]
\]

Hence by Slepian's inequality,
\[
\mathbb{P}(\sup_{t \in [0,T]} f_t > u) \geq
\mathbb{P}(\max_{i \in \{1,\hdots,m\}} f_{t_i} > u) \geq
\mathbb{P}(\max_{i \in \{1,\hdots,m\}} Z_i > u)\]
for all $u > 0$.
Thus, taking $u = (1-\varepsilon)\sqrt{2\log(T+1)}$
so that
\[
u \leq 
\left(1-\frac{\varepsilon}{2}\right) \sqrt{2(1-C(\tau)) \log\left(\frac{T}{\tau}-1\right)}
\]
and
\[
\frac{\sqrt{1-C(\tau)}}{u} - \frac{(1-C(\tau))^{3/2}}{u^3} \leq \frac{\sqrt{1-C(\tau)}}{2u}
\]
we have
\begin{align}
\mathbb{P}(\sup_{t \in [0,T]} f_t < \sqrt{2\log (T-1)}(1-\epsilon))
&\leq \mathbb{P}(\max_{i \in \{1,\hdots,m\}} Z_i < u)
\\&= \left(1-\Psi\left(\frac{u}{\sqrt{1-C(\tau)}}\right)\right)^m
\\&\leq \left(1-\left(\frac{\sqrt{1-C(\tau)}}{u} - \frac{(1-C(\tau))^{3/2}}{u^3}\right)\phi\left(\frac{u}{\sqrt{1-C(\tau)}}\right)\right)^m
\\&\leq \left(1-\left(\frac{\sqrt{1-C(\tau)}}{2u}\right)\phi\left(\frac{u}{\sqrt{1-C(\tau)}}\right)\right)^m
\\& \leq \exp\left(-m\left(\frac{\sqrt{1-C(\tau)}}{2u}\right)\phi\left(\frac{u}{\sqrt{1-C(\tau)}}\right)\right) 
\end{align}

Now note
that as $T \to \infty$,
\[
\mathbb{P}\left(\lim_{T \to \infty}\frac{\max
    Z_t}{\sqrt{2D\log(T)}} = 1
\right) = 1
\]
This suggests, and with some more detailed analysis, implies that
\[
\mathbb{P}\left(\limsup_{T \to \infty}\frac{\sup_{[-T,T]^D}
    f_t}{\sqrt{2D\log(T)}} \geq 1
\right) = 1
\]

\emph{1. Upper bound}.

Partition $[-T,T]^D$ into hypercubes of edge length 1.
By union bound and Borrell-TIS inequality,
\[\mathbb{P}(\sup_{[-T,T]^D} f_t \geq u) \leq (2T)^D
\mathbb{P}(\sup_{[-T,T]^D} f_t \geq u) \leq (2T)^D e^{-u^2/2}\]
This can be used to show
\[
\mathbb{P}\left(\limsup_{T \to \infty}\frac{\sup_{[-T,T]^D}
    f_t}{\sqrt{2D\log(T)}} \leq 1
\right) = 1
\]




\section{References}

Adler RF, Taylor J. \emph{Random Fields and Geometry}

\end{document}
