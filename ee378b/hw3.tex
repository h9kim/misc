\documentclass[11pt]{article} 

% packages with special commands
\usepackage{amssymb, amsmath}
\usepackage{epsfig}
\usepackage{array}
\usepackage{ifthen}
\usepackage{color}
\usepackage{fancyhdr}
\usepackage{graphicx}
\usepackage{mathtools}
%\usepackage{algorithm}
%\usepackage{algpseudocode}
%\usepackage{mdframed}
%\newmdtheoremenv{lem}{Lemma}
\definecolor{grey}{rgb}{0.5,0.5,0.5}

\begin{document}
\newcommand{\tr}{\text{tr}}
\newcommand{\E}{\textbf{E}}
\newcommand{\diag}{\text{diag}}
\newcommand{\argmax}{\text{argmax}}
\newcommand{\Cov}{\text{Cov}}
\newcommand{\Var}{\text{Var}}
\renewcommand{\thefootnote}{\fnsymbol{footnote}}

\begin{center}
\noindent Charles Zheng EE 378b HW 3
\end{center}

\noindent\textbf{1.}
a.
Since the space of unit vectors is compact,
there exists $x^0$ such that $||M||_2 = ||Mx^0||_2$.
For the same reason, there exist unit vectors $x^\circ, y^\circ$ such that
\[
\langle x^\circ, M y^\circ \rangle = \max_{||x|| = ||y|| = 1} \langle x, My \rangle
\]

Letting $y^0 = Mx^0/||M||_2$, we have $||y||_2 = 1$.
Therefore
\[
||M||_2 = (y^0)^* M x^0 \leq \max_{||y||=||x||=1} \langle y, Mx \rangle
\]

Also, by Cauchy-Schwarz we have 
\begin{align*}
\max_{||x|| = ||y|| = 1} \langle x, My \rangle
= (x^\circ)^* M y^\circ
\leq ||x^\circ||_2 ||My^\circ||_2 \leq ||My^\circ||_2 
\leq \max_{||y||=1} ||My||_2 = ||M||_2
\end{align*}

Having shown that \[||M||_2 \leq \max_{||x|| = ||y|| = 1} \langle x, My \rangle \leq ||M||_2\]
we conclude that the definitions are equivalent

b.  Let the SVD of $M$ be written $M = UDV^T$ where $D =
diag(\sigma_1,\hdots, \sigma_n)$.
Let $v_1$ be the first column of $V$, then $||v_1||_2 = 1$ and
\[
||Mv_1||_2 = ||UDV^T v_1||_2 = ||UD e_1||_2 = ||De_1||_2 = \sigma_1
\]
Hence
\[
\sigma_1 = ||Mv_1||_2 \leq \max_{||x|| = 1} ||Mx||_2 = ||M||_2
\]

Meanwhile for any unit vector $x$, defining $y = V^T x$ we have $||y||_2 \leq 1$.
Then
\[
\max_{||x||=1} ||Mx||_2 = \max_{||x|| = 1} ||UDV^T x||_2
\leq \max_{||x|| = 1} ||DV^T x||_2 \leq \max_{||y|| = 1} ||Dy||_2
\]
But defining $a_i = y_i^2$,
\[
\max_{||y||=1} ||Dy||_2^2 = \max_{||y||=1} \sum_{i=1}^n \sigma_i^2 y_i^2
= \max_{\sum a_i = 1, a_i \geq 0} \sum_{i=1}^n \sigma_i^2 a_i
\]
is maximized by $a = e_1$, hence $\max_{||y|| = 1} ||Dy||_2 = \sigma_1$.

Having shown that
\[
\sigma_1 \leq ||M||_2 \leq \sigma_1
\]
we conclude that the two definitions are equivalent.\\

\noindent\textbf{2.}
a.
From 1a we have
\[
||M^*||_2 = \max_{||x||=||y||=1} \langle x, M^* y \rangle
= \max_{||x||=||y||=1} \langle y, M x \rangle = ||M||_2
\]

b.
We have
\[
||AB||_2 = \max_{||x|| = 1} ||ABx||_2 = \max_{y = Bx\text{ for some }||x|| = 1} ||Ay||_2
\]
Meanwhile, if $y = Bx$, and $||x||_2 = 1$, we have $||y||_2 \leq ||B||_2$.
Therefore the set
$\{y: y = Bx \text{ for some }x\text{ such that }||x||=1\}$
is contained in the set
$\{y: ||y||_2 \leq ||B||_2\}$.
Hence
\[
\max_{y = Bx\text{ for some }||x|| = 1} ||Ay||_2 \leq
\max_{||y|| = ||B||_2} ||Ay||_2 = ||A||_2 ||B||_2
\]

\noindent\textbf{3.}
i. 
\[
||aM||_2 = \max_{||x||=1} ||aMx||_2 = \max_{||x||=1} |a| ||M_2 x||_2 = |a|\max_{||x||=1} ||M x||_2 = a||M||_2
\]

ii.
\[
||A + B||_2 = \max_{||x|| = 1} ||Ax + Bx||_2 \leq \max_{||x||=1} ||Ax||_2 + ||Bx||_2
\leq \max_{||x|| = 1} ||Ax||_2 + \max_{||x||=1} ||Bx||_2 = ||A||_2 + ||B||_2
\]

iii.
Proof of contrapositive:
If $M \neq 0$, then some column $M_i$ is nonzero.
But then $||Me_i||_2 = ||M_i||_2 > 0$, so $||M||_2 > 0$.\\


\noindent\textbf{4.}
a.
Let $m_i$ denote the columns of $M^*$.
\[
||M||_2 = ||M^*||_2 = \max_{||x|| = 1} M^* x 
= \left \| \sum_{i=1}^m x_i m_i \right \|
\leq \sum_{i=1}^m x_i ||m_i||_2
\]
Now let $\mu$ be the vector defined by $\mu_i = ||m_i||_2$.
By H\"{o}lder's inequality, we have
\[
||M||_2 \leq \max_{||x||_2= 1} \langle x, \mu \rangle \leq \max_{||x||_2 = 1} ||x||_1 ||\mu||_\infty
\]
But $\max_{||x||_2 = 1} ||x||_1 = \sqrt{m}$ and $||\mu||_\infty = \max_i ||m_i||_2$.
Hence
\[
||M||_2 \leq \sqrt{m} \max_i ||m_i||_2
\]
as needed.

We see that the upper bound is tight by taking $M = 1_m 1_n^T$,
in which case $m_1 = \hdots = m_m = 1_n$ 
\[||M||_2 = ||M \frac{1}{\sqrt{n}}1_n||_2 = ||\sqrt{n} 1_m||_2 = \sqrt{m}||1_n||_2 = \sqrt{m} \max_i ||m_i||_2
\]

b.  Using the fact that $||x||_2 \geq \frac{1}{\sqrt{m}}
||x||_1$ for any vector $x \in \mathbb{R}^m$, we have
\[
||M||_2 \geq ||M\frac{1}{\sqrt{n}} 1_n||_2
= ||\mu||_2 \geq \frac{1}{\sqrt{m}} ||\mu||_1 = \frac{1}{\sqrt{mn}} \sum_{i=1}^m |m_i^* 1_n|
\]
where $\mu$ is the $m$-vector with $m_i = \frac{1}{\sqrt{n}} m_i^* 1_n$.
Hence the upper bound is proved.

To show that the upper bound is tight, take $M = 1_m 1_n^T$, in which case
\[
||M||_2 = \sqrt{mn} = \frac{1}{\sqrt{mn}} mn = \frac{1}{\sqrt{mn}}  \sum_{i=1}^m |m_i^* 1_n|
\]




\noindent\textbf{5.}
a.
We have
\[
||M||_F = \sqrt{\sum_i \sum_j M_{ij}^2} = \sqrt{\tr M^T M} = \sqrt{\tr M^T M}
\]
Without loss of generality assume $n \leq m$.
Since the eigenvalues of $M^T M$ are $\sigma_1^2,\hdots, \sigma_n^2$,
we have
\[
\tr M^T M = \sum_{i=1}^n \sigma_i^2
\]
hence dropping the assumption that $n \leq m$, we have
\[
||M||_F = \sqrt{\sum_{i=1}^{\min(n, m)} \sigma_i^2}
\]

b.
Only $\sigma_1,\hdots, \sigma_r$ are nonzero, where $r = \text{rank}(M)$.
Letting $d = (\sigma_1,\hdots, \sigma_r)$, we have
$||M||_F = ||d||_2$.
But we know that for all vectors $x \in \mathbb{R}^r$,
\[
||x||_\infty \leq ||x||_2 = \sqrt{\sum_{i=1}^r x_i^2} \leq \sqrt{\sum_{i=1}^r ||x||_\infty^2} = \sqrt{r} ||x||_\infty
\]
Hence
\[
||M||_2 = \sigma_1 = ||d||_\infty \leq ||d||_2 = ||M||_F = ||d||_2 \leq \sqrt{r}||d||_\infty = \sqrt{r} ||M||_2  
\]
as needed.

\end{document}
