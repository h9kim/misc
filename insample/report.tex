\documentclass[11pt]{article} 

% packages with special commands
\usepackage{amssymb, amsmath}
\usepackage{epsfig}
\usepackage{array}
\usepackage{ifthen}
\usepackage{color}
%\usepackage{fancyhdr}
\usepackage{graphicx}
\usepackage{indentfirst}
\usepackage{caption}
%\usepackage{mathtools}
\definecolor{grey}{rgb}{0.5,0.5,0.5}

\begin{document}
\newcommand{\tr}{\text{tr}}
\newcommand{\E}{\textbf{E}}
\newcommand{\diag}{\text{diag}}
\newcommand{\argmax}{\text{argmax}}
\newcommand{\argmin}{\text{argmin}}
\newcommand{\Cov}{\text{Cov}}
\newcommand{\Vol}{\text{Vol}}
%\pagestyle{fancy}

\title{Comparing in-sample and out-of-sample error for ridge regression}

\author{Charles Zheng\thanks{with thanks to Lucas Janson and Zhou Fan}}

\maketitle

\section{Introduction}

\subsection{Ordinary least squares}

Consider a linear model where $y = X\beta + \epsilon$, with $\epsilon$
having independent, zero-mean entries, all with the same variance
$\sigma^2$.  We observe $y$ and $X$ and estimate $\beta$ by
$\hat{\beta} = (X^T X)^{-1} X^T y$.  Now consider the problem of
predicting an independent set of observations
\[
y^* = X\beta + \epsilon^*
\]
where $\epsilon^*$ is an independent copy of $\epsilon$ and where the
design matrix $X$ is unchanged from before.  We can predict the values
of $y^*$ by $\hat{y} = X\hat{\beta}$.  Define the \emph{in-sample}
prediction risk by
\[
\text{r}_{in} = \frac{1}{n} \E ||\hat{y} - y^*||^2
\]
where the term ``in-sample'' refers to the design matrix $X$ is the
same for the predicted observations as for the training data.  The
in-sample error $\text{r}_{in}$ can equally well be defined by the
error of prediction for a single observation $y^*$ conditional on
observing its covariate $x^*$, where $x^*$ is drawn uniformly at
random among the $n$ rows of $X$.  It is well-known that under the
previous assumptions,
\[
\text{r}_{in} = \sigma^2 \left(1 + \frac{p}{n}\right)
\]
To prove this fact, note that $\hat{y} = Hy$, where $H$ is the
projection onto the column space of $X$, and that $\tr(H) = p$.  Then
\begin{align*}
\text{r}_{in} &= \frac{1}{n} \E ||y^* - \hat{y}||^2
\\&= \frac{1}{n} ||y^* - Hy||^2
\\&= \frac{1}{n} ||(X\beta + \epsilon^*) - H(X\beta + \epsilon)||^2
\\&= \frac{1}{n} ||(I-H)X\beta + \epsilon^* - H\epsilon||^2
\\&= \frac{1}{n} ||\epsilon^* - H\epsilon||^2 \text{ (since $HX = X$) }
\\&= \frac{1}{n} ||\epsilon||^2 + ||H\epsilon||^2
\\&= \frac{1}{n} \tr(\sigma^2 I) + \tr(\sigma^2 H)
\\&= \frac{1}{n} \sigma^2(n + p)
\end{align*}
which yields the desired formula.

The concept of in-sample error arises naturally in problems where the
design matrix $X$ is fixed, e.g. controlled experiments.  In
observational data it is more natural to suppose that observations
$(x_i, y_i)$ are drawn from some joint distribution $F$.  Supposing we
observe i.i.d. realizations $(x_i,y_i)$ are drawn iid from $F$, then
forming the design matrix $X$ by stacking the $x_i$, we again obtain a
least-squares estimate $\hat{\beta}$ for the coefficients of the best
linear approximation of $y$ conditional on $x$.  Now suppose we obtain
a new independent realization $(x^*, y^*)$ from $F$, but only observe
$x^*$.  As before, we predict $\hat{y} = (x^*)^T \hat{\beta}$, and now
we define the average \emph{out-of-sample} prediction risk by
\[
\text{r}_{out} = \E(\hat{y} - y^*)^2
\]

With suitable assumptions we can derive a similar formula for the
out-of-sample risk.  The following result is due to Lucas Janson.
Suppose that $F$ is a multivariate gaussian with mean 0 and covariance
\[
\Sigma_{xy} = \begin{pmatrix}\Sigma & \Sigma \beta \\
\beta^T \Sigma & \beta^T \Sigma \beta + \sigma^2 \end{pmatrix}
\]
i.e. $x \sim N(0, \Sigma)$ and $y|x \sim N(x^T \beta, \sigma^2)$.
Then using the fact that $(\hat{\beta}- \beta)|X \sim N(0,
\sigma^2(X^T X)^{-1})$ we have
\begin{align*}
\text{r}_{out} &= \E (y^* - \hat{y})^2
\\&= \E (\beta^T x^* + \epsilon^* - \hat{\beta}^T x^*)^2
\\&= \E ((\beta - \hat{\beta})^T x^* + \epsilon^*)^2
\\&= \E((\beta - \hat{\beta})^T x^*)^2  + \E(\epsilon^*)^2
\\&= \sigma^2 + \E((\beta - \hat{\beta})^T x^*)^2
\\&= \sigma^2 + \tr\E(x^* (x^*)^T (\beta - \hat{\beta})(\beta - \hat{\beta})^T)
\intertext{using independence of $(X,y)$ and $x^*$,}
&= \sigma^2 + \tr[\E[x^* (x^*)^T]\E[(\beta - \hat{\beta})(\beta - \hat{\beta})^T]]
\\&= \sigma^2 + \tr[\Sigma \E[(\beta - \hat{\beta})(\beta - \hat{\beta})^T]]
\\&= \sigma^2 + \tr\E[\Sigma \E[(\beta - \hat{\beta})(\beta - \hat{\beta})^T|X]]
\\&= \sigma^2 + \E[\tr[\Sigma (\sigma^2 (X^T X)^{-1})]]]
\\&= \sigma^2 + \sigma^2 \E[\tr[\Sigma^{1/2} ((X^T X)^{-1}) \Sigma^{1/2}]]]
\end{align*}
Note that $\Sigma^{1/2} (X^T X)^{-1} \Sigma^{1/2}$ has an
inverse-Wishart distribution with identity scale matrix and $n$
degrees of freedom.  Hence
\[
\E[\tr[\Sigma^{1/2} ((X^T X)^{-1}) \Sigma^{1/2}]]] = \frac{p}{n-p-1}
\]
and thus
\[
\text{r}_{out} = \sigma^2\left(1 + \frac{p}{n-p-1}\right)
\]
Comparing with $\text{r}_{in}$, we see that $\text{r}_{out}$ is
strictly larger, since $p/n$ has been replaced by $p/(n-p-1)$.

\subsection{Ridge regression}

We see in the OLS case that out-of-sample risk is greater than
in-sample risk, with the difference becoming more and more pronounced
as $p$ increases relative to $n$.  Hence it is especially interesting
to consider the relationship between out-of-sample risk and in-sample
risk in an extremely high-dimensional setting.  Of course, since OLS
cannot be applied when $p > n$, we could only derive the formulas for
a method such as ridge regression.

Ridge regression can be used to estimate a linear model when $p > n$
by using the estimator
\[
\hat{\beta}_\lambda = (X^T X + n\lambda)^{-1} X^T y
\]
where $\lambda > 0$ is a regularization parameter.

Dobriban and Wager (2015) obtain asymptotic expressions for
$\text{r}_{out}$ of ridge regression; using similar methods, we obtain expressions for the in-sample error $\text{r}_{in}$.

Dobriban and Wager (2015) consider a sequence of multivariate normal
models for $(x,y)$, but in which $\beta$ is also a random variate, and
in an asymptotic regime where both $p$ and $n$ grow to infinity,
approaching a ratio $\gamma = p/n$.  Since $p$ is changing, the
covariance matrix $\Sigma_p$ must be different for each model in the
sequence, but one assumes that the distribution of eigenvalues of
$\Sigma_p$ converges in distribution to a limiting eigenvalue
distribution $H(\lambda)$ on the real line.  Meanwhile, it is assumed
that $\beta \sim N(0, \frac{\alpha^2\sigma^2}{p} I)$ so that $\frac{||\beta||^2}{\sigma^2}$ approaches a constant $\alpha^2$.  It is
shown that under such a setup, the asymptotically
the optimal value of $\lambda$ is given by
\[
\lambda^* = \frac{\gamma}{\alpha^2}
\]
and using this value of $\lambda$, one obtains
\[
\text{r}_{out} = \E(y^* - \hat{\beta}_{\lambda^*}^T x^*)^2 =  \sigma^2\left(\frac{1}{\lambda^* v_{H, \gamma}(-\lambda^*)}\right)
\]
where $v_{H,\gamma}$ will be defined below.

Using similar methods we derive an expression for $\text{r}_{in}$.
The key fact from random matrix theory we use is that if
$\hat{\Sigma}_p$ is the empirical covariance matrix for a sequence of
distributions $N(0, \Sigma_p)$ where $\Sigma_p$ have limiting spectrum
$H(\lambda)$, then
\[
\lim \frac{1}{p} \tr((\hat{\Sigma}_p - zI_{p \times p})) = m_{H, \gamma}(z)
\]
for all $z \in \mathbb{C}\setminus \mathbb{R}^+$,
where $m_{H, \gamma}(z)$ is a well-known function from random matrix theory, which can be computed for distribution $H$ from the fixed-point formula
\[
m_H(z) = \int_{t=0}^\infty \frac{dH(t)}{t(1- \gamma - \gamma z mz(z)) - z}
\]
which is known as the Marchenko-Pasture formula, or Silverstein formula.
Meanwhile, the function $v_{H,\gamma}$ appearing in the out-of-sample risk formula is related to $m_{H,\gamma}$ by
\[
\gamma( m(z) + 1/z) = v(z) + 1/z
\]
The limit
\[
\lim \frac{1}{p} \tr((\hat{\Sigma}_p - zI_{p \times p})) = m_{H, \gamma}(z)
\]
can also be expressed as
\[
\lim \frac{1}{p} \sum_{i=1}^p \frac{1}{\lambda_i - z} \to m_{H, \gamma}(z)
\]
where $\lambda_i$ are the sample eigenvalues.
Hence we also have
\begin{align*}
\lim \frac{1}{p} \tr(\hat{\Sigma}(\hat{\Sigma}_p - zI_{p \times p}))
&= \lim \frac{1}{p} \sum_{i=1}^p \frac{\lambda_i}{\lambda_i - z}
\\&= \lim \frac{1}{p} \sum_{i=1}^p \left( 1 + \frac{z}{\lambda_i - z}\right)
\\&= \lim 1 + z\frac{1}{p} \sum_{i=1}^p \frac{1}{\lambda_i - z}
\\&= 1 + zm_{H,\gamma}(z)
\end{align*}



Our result is as follows.  
Note that
\begin{align*}
\hat{\beta}_\lambda - \beta &= (X^T X + n\lambda I)^{-1} X^T y 
\\&= ((\hat{\Sigma} + \lambda I)^{-1}\hat{\Sigma} - I)\beta + \frac{1}{n}(\hat{\Sigma} + \lambda I)^{-1} X^T \epsilon
\\&= (\hat{\Sigma} + \lambda I)^{-1} \left(-\lambda \beta + \frac{1}{n} X^T \epsilon\right)
\end{align*}

For $y^* = X \beta + \epsilon^*$ where
$\epsilon^*$ is an independent copy of $\epsilon$, we have (as $n, p
\to \infty$)
\begin{align*}
\text{r}_{in} \stackrel{def}{=}& \frac{1}{n}\E||y^* - X\hat{\beta}_{\lambda^*}||^2
\\=& \frac{1}{n}\E||X\beta + \epsilon^* - X\hat{\beta}_{\lambda^*}||^2
\\=& \sigma^2 + \frac{1}{n}\E||X(\beta - \hat{\beta}_{\lambda^*})||^2
\\=& \sigma^2 + \frac{1}{n} \E (\beta - \hat{\beta}_{\lambda^*})^T X^T X (\beta - \hat{\beta}_{\lambda^*})
\\=& \sigma^2 + \E (\beta - \hat{\beta}_{\lambda^*})^T \hat{\Sigma} (\beta - \hat{\beta}_{\lambda^*})
\\=& \sigma^2 + \E (X^T\epsilon/n - \lambda \beta)^T (\hat{\Sigma} + \lambda I)^{-1} \hat{\Sigma} (\hat{\Sigma} + \lambda I)^{-1} (X^T\epsilon/n - \lambda \beta))
\\=& \sigma^2 + (1/n^2)\E [\epsilon^T X (\hat{\Sigma} + \lambda I)^{-1} \hat{\Sigma} (\hat{\Sigma} + \lambda I)^{-1} X^T \epsilon]
\\&+ \lambda^{*2} \E[ \beta^T (\hat{\Sigma} + \lambda I)^{-1} \hat{\Sigma} (\hat{\Sigma} + \lambda I)^{-1} \beta]
\\=& \sigma^2 + (\sigma^2/n)\tr \E [\hat{\Sigma}(\hat{\Sigma} + \lambda I)^{-1} \hat{\Sigma} (\hat{\Sigma} + \lambda I)^{-1}]
\\&+ \lambda^{*2} \frac{\alpha^2\sigma^2}{p}\tr\E[(\hat{\Sigma} + \lambda I)^{-1} \hat{\Sigma} (\hat{\Sigma} + \lambda I)^{-1}]
\\&= \frac{\sigma^2}{n}\left[\tr\E [\hat{\Sigma}(\hat{\Sigma} + \lambda I)^{-1} \hat{\Sigma} (\hat{\Sigma} + \lambda I)^{-1}]
 + \lambda^* \tr\E[\hat{\Sigma}(\hat{\Sigma} + \lambda I)^{-2}  ]\right]
\\&= \sigma^2 + \frac{\sigma^2}{n} \tr\E [\hat{\Sigma}(\hat{\Sigma} + \lambda I)^{-1}]
\end{align*}
where in the last line we used
\[
\hat{\Sigma}(\hat{\Sigma} + \lambda I)^{-1} \hat{\Sigma} (\hat{\Sigma} + \lambda I)^{-1} = \hat{\Sigma}(\hat{\Sigma} + \lambda I)^{-1}  - \lambda \hat{\Sigma}(\hat{\Sigma} + \lambda I)^{-2}
\]
Thus
\[
\lim_{n \to \infty} \text{r}_{in} = \sigma^2 \left(1 + \lim_{n \to \infty} \frac{\gamma}{p} \tr\E [\hat{\Sigma}(\hat{\Sigma} + \lambda^* I)^{-1}]\right) = \sigma^2 (1 + \gamma (1- \lambda^* m(-\lambda^*)))
\]

TODO:
\begin{itemize}
\item These formulae have been confirmed numerically.  Todo: include the plots and tables
\item Interpret the formulae for special cases, e.g. identity covariance and AR-1 covariance
\item From these formulae, derive a simpler formula for the relationship of out-of-sample to in-sample risk
\end{itemize}

\end{document}
